\section*{Introduction}
This project paper is a part of a bigger project goal, spanning over several different implementations, which is to map a maze. By using different sensor technologies and implementation platforms, we are attempting to extract the characteristics of a maze, that is the length of the walls, the position and the orientation. \\

The purpose of this project is to improve the characterization and measurement of a maze by using and implementing computer vision on an unmanned aerial vehicle flying above the maze. By using characteristics of an image sensor and information from a height sensor and the UAV's position, it is possible to determine the physical dimensions and characteristics of an object, in this case a maze, on the ground. The goal is to assist the Lego robots in the mapping of the maze, by supplying an improved estimate of the dimensions and the position of the maze. \\

Traditionally, the maze has been characterized using Lego Robots with SLAM implementation, but the sensor used to determine the position of the walls has a margin of error that increases the longer the robots run. So in order to assist with the characterization of the maze, and thus eliminating some of the margin of error, the project explores how the use of an unmanned aerial vehicle with an image sensor can assist in the mapping. There are many factors that determine the accuracy of determining the dimensions and location of the maze by using computer vision, and the most important of these are presented in this report.